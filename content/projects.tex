\section*{Projekte}
\renewcommand{\arraystretch}{1.3}
\begin{longtable}{@{}>{}p{4cm}>{\itshape}p{2cm}>{}p{9cm}}
12/2015             & Kunde 	    & Internet of Things Anwendungshersteller\\
\nopagebreak		& Thematik	    & Containerisierung einer Microservice-Architektur und Anpassung für den Betrieb in einem Kubernetes Cluster sowie dessen Installation\\
\nopagebreak		& Rolle 	    & Software Architekt, Docker Spezialist\\
\nopagebreak		& Technologien	& AWS, Docker, Kubernetes, Java, Postgres, MongoDB\\
\\
05/2015-12/2015     & Kunde 	    & Fernsehübertragung\\
\nopagebreak		& Thematik	    & Entwicklung einer Playout-Software für die Übertragung von Videos auf TV-Satelliten\\
\nopagebreak		& Rolle 	    & Software Architekt, Entwickler\\
\nopagebreak		& Technologien	& Go, Docker, Puppet, Snmp, Prtg, Fluend, Graylog, IntelliJ IDEA, Git\\
\\
08/2014-12/2015     & Kunde 	    & (in-house)\\
\nopagebreak		& Thematik	    & Entwicklung einer Schulungsanwendung für den Continuous Delivery Prozess von Microservice Architekturen\\
\nopagebreak		& Rolle 	    & Software Architekt, Entwickler\\
\nopagebreak		& Technologien	& Docker, Docker-Compose, Kubernetes, Mesos, MongoDB, Redis, AngularJS, Apache Wicket, Scala, Groovy, IntelliJ IDEA, Git\\
\\
03/2015             & Kunde 	    & Krankenversicherung\\
\nopagebreak		& Thematik	    & Docker Schulung – Basics und Cluster-Management\\
\nopagebreak		& Rolle 	    & Coach\\
\nopagebreak		& Technologien	& Docker, Docker-Compose, Kubernetes\\
\\
08/2014-03/2015     & Kunde 	    & Urlaubsbewertungsportal\\
\nopagebreak		& Thematik	    & Entwicklung eines Microservice Architektur Ansatz zur Migration eines Urlaubsbewertungsportal (POC Integration des CMS Prismic.io) und Wartung der mobilen Variante des Portals\\
\nopagebreak		& Rolle 	    & Software Architekt, Entwickler\\
\nopagebreak		& Technologien	& Docker, Mesos, MongoDB, NodeJS, Scala, Prismic, IntelliJ IDEA, Git\\
\\
11/2013-08/2014     & Kunde 	    & Mobilfunkbetreiber\\
\nopagebreak		& Thematik	    & Wartung eines Onlineshops und Neuentwicklung der mobilen Variante, sowie Auffinden von Performance Bottlenecks und Memory Leaks und deren Beseitigung\\
\nopagebreak		& Rolle 	    & Software Architekt, Entwickler\\
\nopagebreak		& Technologien	& Oracle, Spring, Hibernate, Maven, Apache Wicket, IntelliJ IDEA, Git\\
\\
10/2013             & Kunde 	    & Einzelhandel\\
\nopagebreak		& Thematik	    & Prototyp für die Migration eines Multifaktsterns in Hadoop\\
\nopagebreak		& Rolle 	    & Software Architekt, Entwickler\\
\nopagebreak		& Technologien	& Postgres, Hadoop, Spring, Gradle, IntelliJ IDEA, Git-SVN\\
\\
06/2013             & Kunde 	    & (in-house)\\
\nopagebreak		& Thematik	    & Entwicklung eines Prototyps für eine Recommendation-Engine\\
\nopagebreak		& Rolle 	    & Scrum Master, Software Architekt, Entwickler\\
\nopagebreak		& Technologien	& MongoDB, Spring, Spring Data, Mahout, Gradle, Apache Wicket, IntelliJ IDEA, Git\\
\\
04/2013-11/2013     & Kunde 	    & Einzelhandel\\
\nopagebreak		& Thematik	    & Integration der Reporting Lösung Webfocus in einen Report-Bestellprozess in einem Liferay Portal\\
\nopagebreak		& Rolle 	    & Software Architekt, Entwickler\\
\nopagebreak		& Technologien	& Postgres, Greenplum, Webfocus, Spring, Gradle, IntelliJ IDEA, Git-SVN, Liferay\\
\\
06/2012             & Kunde 	    & Mobilfunkbetreiber\\
\nopagebreak		& Thematik	    & Prototyp für die Realtime Visualisierung von Bewegungsdaten \\
\nopagebreak		& Rolle 	    & Software Architekt, Entwickler\\
\nopagebreak		& Technologien	& MongoDB, Hadoop, Spring, Maven, BackboneJS, IntelliJ IDEA, Git\\
\\
03/2011-12/2012     & Kunde 	    & Mobilfunkbetreiber\\
\nopagebreak		& Thematik	    & Neuentwicklung eines Onlineshops \\
\nopagebreak		& Rolle 	    & Software Architekt, Entwickler\\
\nopagebreak		& Technologien	& Oracle, Spring, Hibernate, Maven, Apache Wicket, IntelliJ IDEA, Git\\
\\
02/2011             & Kunde 	    & Versicherung\\
\nopagebreak		& Thematik	    & Entwicklung eines Intranetportals / CMS\\
\nopagebreak		& Rolle 	    & Software Architekt, Entwickler\\
\nopagebreak		& Technologien	& PHP, Drupal, MySQL, Eclipse, Subversion\\
\\
12/2010-02/2011     & Kunde 	    & Tourismus\\
\nopagebreak		& Thematik	    & Entwicklung einer Auktionsplattform für Reisen\\
\nopagebreak		& Rolle 	    & Software Architekt, Entwickler\\
\nopagebreak		& Technologien	& PHP, XSL, MySQL, Eclipse, Subversion\\
\\
12/2010         	& Kunde 	    & Pharmazie\\
\nopagebreak		& Thematik	    & Entwicklung einer Webservice-Mittelschicht, um eine interne Anwendung für Dienstreisenbuchungen mit einem externen Dienstleister zu verbinden\\
\nopagebreak		& Rolle 	    & Software Architekt, Entwickler\\
\nopagebreak		& Technologien	& PHP, XSL, MySQL, SOAP, Eclipse, Subversion\\
\\
08/2010 - 11/2010	& Kunde 	    & Verlagswesen\\
\nopagebreak		& Thematik	    & Entwicklung eines Portals für die Suche von Unternehmen\\
\nopagebreak		& Rolle 	    & Software Architekt, Entwickler\\
\nopagebreak		& Technologien	& PHP, Solr, PostgreSQL, PHP. XSL, Eclipse, Subversion\\
\\
09/2009 - 08/2010	& Kunde 	    & (in-house)\\
\nopagebreak		& Thematik	    & Einführung eines ERP-Systems mit agilen Methoden\\
\nopagebreak		& Rolle 	    & Diplomand\\
\nopagebreak		& Technologien	& ARIS, BIRT, Exchange, LDAP, Postgres, Projektron BCS, UML, Eclipse, Subversion\\
\\
04/2009 - 06/2009	& Kunde 	    & Human-Ressource\\
\nopagebreak		& Thematik	    & Integration einer Job-Suchmaschine in ein Typo3 Portal\\
\nopagebreak		& Rolle 	    & Werkstudent\\
\nopagebreak		& Technologien	& PHP, Typo3, Lucene, Hibernate, Maven, Eclipse, Subversion\\
\\
11/2008 - 03/2009	& Kunde 	& (in-house)\\
\nopagebreak		& Thematik	& Implementierung einer semantischen Job-Suchmaschine\\
\nopagebreak		& Rolle 	& Werkstudent\\
\nopagebreak		& Technologien	& Lucene, Maven, MySQL, Eclipse, Subversion\\
\\
05/2008 - 10/2008	& Kunde 	& Verlagswesen\\
\nopagebreak		& Thematik	& Implementierung von Porlets für ein Liferay Portal\\
\nopagebreak		& Rolle 	& Werkstudent\\
\nopagebreak		& Technologien	& Liferay, MySQL, Eclipse, Subversion\\
\end{longtable}
\renewcommand{\arraystretch}{2}


